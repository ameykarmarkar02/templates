% Options for packages loaded elsewhere
\PassOptionsToPackage{unicode}{hyperref}
\PassOptionsToPackage{hyphens}{url}
%
\documentclass[]{article}
\usepackage{amsmath,amssymb}
\usepackage{lmodern}
\usepackage{iftex}
\usepackage{graphicx} % Required for inserting images
\usepackage[margin=72pt]{geometry}

\ifPDFTeX
  \usepackage[T1]{fontenc}
  \usepackage[utf8]{inputenc}
  \usepackage{textcomp} % provide euro and other symbols
\else % if luatex or xetex
  \usepackage{unicode-math}
  \defaultfontfeatures{Scale=MatchLowercase}
  \defaultfontfeatures[\rmfamily]{Ligatures=TeX,Scale=1}
\fi
% Use upquote if available, for straight quotes in verbatim environments
\IfFileExists{upquote.sty}{\usepackage{upquote}}{}
\IfFileExists{microtype.sty}{% use microtype if available
  \usepackage[]{microtype}
  \UseMicrotypeSet[protrusion]{basicmath} % disable protrusion for tt fonts
}{}
\makeatletter
\@ifundefined{KOMAClassName}{% if non-KOMA class
  \IfFileExists{parskip.sty}{%
    \usepackage{parskip}
  }{% else
    \setlength{\parindent}{0pt}
    \setlength{\parskip}{6pt plus 2pt minus 1pt}}
}{% if KOMA class
  \KOMAoptions{parskip=half}}
\makeatother
\usepackage{xcolor}
\setlength{\emergencystretch}{3em} % prevent overfull lines
\providecommand{\tightlist}{%
  \setlength{\itemsep}{0pt}\setlength{\parskip}{0pt}}
\ifLuaTeX
  \usepackage{selnolig}  % disable illegal ligatures
\fi
\IfFileExists{bookmark.sty}{\usepackage{bookmark}}{\usepackage{hyperref}}
\IfFileExists{xurl.sty}{\usepackage{xurl}}{} % add URL line breaks if available
\urlstyle{same} % disable monospaced font for URLs


\title{Project Final Report for XXX TEAM NUMBER}
\author{Team Member: XXX First and Last name SN1234567}

\begin{document}
\maketitle

\section{Overview}

Provide identification and details on who is on the team and who the
client is. Does the description include a high-level overview about the
specific approach taken in this project along with presenting key system
requirements? What is the UVP that is being offered by the project, who
are the user groups and what is needed by each user group (just don't
put in your use case diagram; this should be a summary in high-level
language of who each user group is and what they are offered by the
product)? From reading this, the reader should have a solid and complete
high-level understanding of the product. \textbf{(2-3 pages including
diagrams)}

\section{System Architecture}

What is the final architecture of the system and what is the tech stack
that was used (make sure to explain why)? What are the major components
of the system design and how is it operationalized? You need to describe
the structure of your project in sufficient detail (i.e. monolithic vs
micro-services; what is the responsibility of each component)? This
needs to include a system architecture diagram, a Level 1 DFD along with
a discussion of tech stack. Be consistent in the use of notation (make
sure to use proper industry-accepted formats and explain each diagram
clearly; you will lose marks if it is just a lonely diagram without a
proper discussion. This section must provide enough detail for a
technical person to understand how the system was built, what the
components are, what are the key processes in the system and how data
flows within the system and is processed. \textbf{(2-3 pages including
diagrams)}

\section{System Features}

Which features are complete and working? A complete list of features
needs to be included to describe the system (\textbf{this part (the
feature list) can be worked on as a team since features may have been
divided up in a way that only one person worked on each feature in the
list}). With this, it needs to include an accurate enumeration of the
features that are working, buggy and what is not working about them.
This should map to the scope of the project (presented in the overview).

Individually, enumerate each feature included in this project as a set
of bullet points, including buggy ones which you will indicate and
explain. Provide 2-3 (brief) sentences to explain each feature so the
reader is clear on how much work goes into making that feature work.
\textbf{Page length will depend on how many features were implemented.}

\begin{enumerate}
    \item Example Feature 1: some explanation about this feature so the reader
knows the time involved and complexity of getting this to work. Lorem
ipsum dolor sit amet, consectetuer adipiscing elit. Ut purus elit,
vestibulum ut, placerat ac, adipiscing vitae, felis. Curabitur dictum
gravida mauris. Nam arcu libero, nonummy eget, consectetuer id,
vulputate a, magna.
\item Example Feature 2: some explanation about this feature so the reader
knows the time involved and complexity of getting this to work. If this
feature is yours, add an Asterisk. Lorem ipsum dolor sit amet,
consectetuer adipiscing elit. Ut purus elit, vestibulum ut, placerat ac,
adipiscing vitae, felis. Curabitur dictum gravida mauris. Nam arcu
libero, nonummy eget, consectetuer id, vulputate a, magna.
\end{enumerate} 

\section{Features You Individually Completed}

How much of the project did you work on? In your report, you will need
to individually provide an annotation of the full feature list AND
clearly indicate which features you have worked on (For example, put an
asterisk beside the feature you did so you don\textquotesingle t re-type
all your features again, or put it into a table). \textbf{For each
feature that you developed, you need to provide enough details for a
technical person to understand how much work goes into building them and
how they work (what does it do/how does it work).} The portion of
features you worked on represents roughly 1/Nth of the work involved in
the project with respect to both quantity and complexity, where N is the
number of members in your team (and should be reflected by repo
activities). \textbf{(Page length will depend on how many features you
implemented, but likely no more than 5 pages)}

\section{Installation and Setup}

How should someone else install and run your project to test it or to
continue development with it? Each team member needs to detail (in their
own words), a clear set of steps to install your project on a local
machine to get it to run successfully. If there are specific environment
settings, versions, or IDE requirements we need to know about, state
those clearly. Note: Do not copy text from the README of your repo
because if multiple students do that, you will all have identical
writing, and your reports will be flagged for plagiarism. This section
should be no more than 1 page.

\section{Lesson's Learned and Project Reflections}

Individually, students will critically evaluate their project
experience, focusing on the development process, challenges encountered,
and personal and team growth. Individuals will reflect on how well the
project objectives and deliverables were met, analyze major challenges
and solutions, and identify key lessons learned. You will need to
reflect on:

\begin{enumerate}
\def\labelenumi{\roman{enumi}.}
\item
  Development Process (considering an analysis of the development
  process, including planning, implementation, and iteration; reflects
  on team dynamics, workflow, and the use of tools and methodologies).
\item
  Challenges and Solutions (reflecting on how well the project
  deliverables met the initial requirements and expectations; discuss
  any deviations and their reasons)
\item
  Personal and Team Growth (\textbf{reflect} on personal and team
  growth, including skills developed, knowledge gained, and how
  experiences will influence future projects)
\item
  Lessons Learned (clearly articulate key lessons that \textbf{YOU}
  learned from the project; reflect on how these lessons will impact
  future work and personal development.
\end{enumerate}

With this reflection:

\begin{itemize}
\item
  \textbf{Be Honest and Reflective:} This exercise is for your personal
  and professional growth. Honest reflections will help you learn from
  your experiences.
\item
  \textbf{Provide Specific Examples:} Use specific examples from your
  project to illustrate your points. This adds depth and credibility to
  your reflections.
\item
  \textbf{Connect to Future Work:} Think about how what you learned
  during this project will influence your future projects and
  professional behavior.
\item
  \textbf{Collaborate and Discuss:} Discuss your reflections with your
  team members. Different perspectives can help you see things you might
  have missed.
\end{itemize}

This section should be no more than 1 page.

\end{document}
